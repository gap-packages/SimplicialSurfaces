% This document contains the TikZ-header for all our LaTeX-computations.
% It especially contains all global graphic parameters.

\usepackage{amsmath, amssymb, amsfonts} % Standard Math-stuff

\usepackage{tikz}
\usetikzlibrary{calc}
\usetikzlibrary{positioning}


% Now we define the global styles

% We start by defining the default colours of vertices, edges and faces
\newcommand{\vertexColor}{\colorVertex}
\newcommand{\edgeColor}{\colorEdge}
\newcommand{\faceColor}{\colorFace}
% If we want to shade a surface with two colours, we need an alternative
\newcommand{\faceColorDark}{\colorDarkFace}

\newcommand{\vSize}{2pt}    % How big are the vertex circles (if drawn)?


\tikzset{vertex/.style = {\vertexColor}}
\tikzset{edge/.style = {\edgeColor, thick}}
\tikzset{edgeBackground/.style = {fill=\colorEdgeBack}}
\tikzset{face/.style = {fill=\faceColor, draw=\edgeColor}}
% The face style for the alternative colouring. The colour changes if
% \differentColors is defined.
\tikzset{faceDark/.style = {fill=\faceColorDark, draw=\edgeColor}}


% Custom command to draw an edge between two points
% First two parameters are the names of the points (no brackets)
% Third parameter is the name of the edge
\newcommand{\drawEdge}[3]{
    \node[edgeBackground] at ($1/2*(#1)+1/2*(#2)$) {#3};
}

% Custom command to draw a bi-arrow with a sharp bend
% Optional: If you want to have a bi-arrow between planes, this is the (Back)-point
% First parameter: Draw options
% Second parameter: Start point
% Third parameter: Middle point
% Fourth parameter: Final point
\newcommand<>{\planEdge}[5][(0,0)]{
    \uncover#6{
        \draw[#2,<-] ($0.5*#1+#3$) -- ($0.5*#1+#4$);
        \draw[#2,->] ($0.5*#1+#4$) -- ($0.5*#1+#5$);
    }
}

\newcommand{\planeOp}{0.8}

% Draw a plane behind the second and third parameters (defined by first one)
% Optinal argument: colour
\newcommand<>{\behindPlane}[4][black]{
    \uncover#5{
        \filldraw[fill=#1!20!white,draw=#1!50!white,opacity=\planeOp] 
            (#3) -- (#4) -- ($(#4)+(#2)$) -- ($(#3)+(#2)$) -- cycle;
        \draw[#1] (#3) -- (#4);
    }
}

% 1: colour (default black)
% 2: movement backwards
% 3: first vertex
% 4: second vertex
% 5: bend angle
\newcommand<>{\curvedPlane}[5][black]{
    \uncover#6{
        \filldraw[fill=#1!20!white,draw=#1!50!white,opacity=\planeOp]
            (#3) to[bend left=#5] (#4) -- ($(#4)+(#2)$) to[bend right=#5] ($(#3)+(#2)$) -- cycle;
        \draw[#1] (#3) to[bend left=#5] (#4);
    }
}
    
\tikzset{graphVertex/.style={circle,fill=gray!70!white,inner sep=1.5pt}}

% Sometimes we want to implement different behaviour for the generated 
% HTML-pictures (for example, shading is not supported in HTML).
% For that we define a macro to check whether we run the code with
% htlatex. The code comes from 
% https://tex.stackexchange.com/questions/93852/what-is-the-correct-way-to-check-for-latex-pdflatex-and-html-in-the-same-latex
\makeatletter
\edef\texforht{TT\noexpand\fi
  \@ifpackageloaded{tex4ht}
    {\noexpand\iftrue}
    {\noexpand\iffalse}}
\makeatother


\usepackage{hyperref}
