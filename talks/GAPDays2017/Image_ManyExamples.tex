                    \begin{tikzpicture}
                        % We need to define the drawing style
	
		        % First a tetrahedron
		        \begin{scope}[xshift=0cm]
			    \coordinate (A) at (0,0);
			    \coordinate (B) at (2,0);
		    	    \coordinate (C) at (0.8,1.5);
			    \coordinate (D) at (1.9,0.7);
			
			    \filldraw[face] (A) -- (B) -- (C) -- cycle;
			    \filldraw[face] (B) -- (C) -- (D) -- cycle;
			    \draw[dashed] (A) -- (D);
		        \end{scope}
		
		        % Second: four triangles in the form of a cone
		        \begin{scope}[xshift=3cm]
			    \coordinate (A) at (0,0);
			    \coordinate (B) at (1.7,0.5);
			    \coordinate (C) at (1.3,1.4);
			    \coordinate (D) at (0.5,1.5);
			    \coordinate (E) at (1,0.7);
			
			    % Take care to draw the faces in the back first
			    \filldraw[faceDark] (A) -- (B) -- (C) -- cycle;
			    \filldraw[faceDark] (A) -- (C) -- (D) -- cycle;
			    % Now the faces in the front
			    \filldraw[face] (A) -- (B) -- (E) -- cycle;
			    \filldraw[face] (A) -- (E) -- (D) -- cycle;
			    % Finally the dashed line for the "hidden" edge
			    \draw[dashed] (A) -- (C);
		        \end{scope}
		
		        % Three triangles that share an edge
		        \begin{scope}[xshift=6cm]
                                    \coordinate (A) at (0,0);
                            \coordinate (B) at (0,1.5);
                            \coordinate (C) at (-0.7,0.4);
                            \coordinate (D) at (0.8,0.4);
                            \coordinate (E) at (0.9,0.8);
			
                            % draw back face first
                            \filldraw[face] (A) -- (B) -- (E) -- cycle;
                            % Now draw front faces
                            \filldraw[face] (A) -- (B) -- (C) -- cycle;
                            \filldraw[face] (A) -- (B) -- (D) -- cycle;
                            % Draw dashed line
                            \draw[dashed] (A) -- (E);


                    	\end{scope}
		
		        % A butterfly of triangles
		        \begin{scope}[xshift=9cm]
                            \def\LUX{-0.8}
                            \def\LUY{-0.3}
                            \def\LMX{-1.2}
                            \def\LMY{0.5}
                            \def\LOX{-0.5}
                            \def\LOY{1}
                            \coordinate (A) at (0,0);
                            \coordinate (B) at (\LUX,\LUY);
                            \coordinate (C) at (\LMX,\LMY);
                            \coordinate (D) at (\LOX,\LOY);
                            \coordinate (E) at (-\LOX,\LOY);
                            \coordinate (F) at (-\LMX,\LMY);
                            \coordinate (G) at (-\LUX,\LUY);
			
                            \filldraw[face] (A) -- (B) -- (C) -- cycle;
                            \filldraw[face] (A) -- (C) -- (D) -- cycle;
                            \filldraw[face] (A) -- (E) -- (F) -- cycle;
                            \filldraw[face] (A) -- (F) -- (G) -- cycle;
		        \end{scope}
		
		        % An open cone of two triangles
		        \begin{scope}[xshift=1cm, yshift=-3cm]
			    \coordinate (A) at (0,0);
			    \coordinate (B) at (1.3,0.4);
			    \coordinate (C) at (0.4,1.3);
			
			    \filldraw[face] (A) -- (B) to[bend right=45] (C) -- cycle;
			    \filldraw[face] (A) -- (B) to[bend left=45] (C) -- cycle;
		        \end{scope}
		
		        % A surface from non-triangular shapes
		        \begin{scope}[xshift=4cm, yshift=-3cm]
			    \coordinate (A) at (0,0);
			    \coordinate (B) at (1,0);
			    \coordinate (C) at (0.5,0.6);
			    \coordinate (D) at (0,1);
			    \coordinate (E) at (0.5,1.3);
			    \coordinate (F) at (1,1);
			    \coordinate (G) at (1.7,0.8);
			    \coordinate (H) at (1.8,0.2);
			
			    \filldraw[face] (A) -- (B) -- (C) -- cycle;
			    \filldraw[face] (A) -- (C) -- (D) -- cycle;
			    \filldraw[face] (D) -- (C) -- (F) -- (E) -- cycle;
			    \filldraw[face] (C) -- (F) -- (G) -- (H) -- (B) -- cycle;
		        \end{scope}

                        % Double tetrahedron
                        \begin{scope}[xshift=8.7cm, yshift=-2.3cm, scale=0.5]
                            \begin{tikzpicture}[vertexBall, edgeDouble, faceStyle, scale=2]

% Define the coordinates of the vertices
\coordinate (V1_1) at (0., 0.);
\coordinate (V2_1) at (1., 0.);
\coordinate (V3_1) at (-0.5, 0.8660254037844384);
\coordinate (V3_2) at (1.5, 0.8660254037844388);
\coordinate (V4_1) at (0.4999999999999999, 0.8660254037844386);
\coordinate (V5_1) at (0.5000000000000001, -0.8660254037844386);
\coordinate (V5_2) at (-0.9999999999999996, 0.);
\coordinate (V5_3) at (2., 0.);


% Fill in the faces
\fill[face]  (V2_1) -- (V4_1) -- (V1_1) -- cycle;
\node[faceLabel] at (barycentric cs:V2_1=1,V4_1=1,V1_1=1) {$1$};
\fill[face]  (V2_1) -- (V3_2) -- (V4_1) -- cycle;
\node[faceLabel] at (barycentric cs:V2_1=1,V3_2=1,V4_1=1) {$2$};
\fill[face]  (V4_1) -- (V3_1) -- (V1_1) -- cycle;
\node[faceLabel] at (barycentric cs:V4_1=1,V3_1=1,V1_1=1) {$3$};
\fill[face]  (V1_1) -- (V5_1) -- (V2_1) -- cycle;
\node[faceLabel] at (barycentric cs:V1_1=1,V5_1=1,V2_1=1) {$4$};
\fill[face]  (V3_1) -- (V5_2) -- (V1_1) -- cycle;
\node[faceLabel] at (barycentric cs:V3_1=1,V5_2=1,V1_1=1) {$5$};
\fill[face]  (V2_1) -- (V5_3) -- (V3_2) -- cycle;
\node[faceLabel] at (barycentric cs:V2_1=1,V5_3=1,V3_2=1) {$6$};


% Draw the edges
\draw[edge] (V2_1) -- node[edgeLabel] {$1$} (V1_1);
\draw[edge] (V1_1) -- node[edgeLabel] {$2$} (V3_1);
\draw[edge] (V1_1) -- node[edgeLabel] {$3$} (V4_1);
\draw[edge] (V5_1) -- node[edgeLabel] {$4$} (V1_1);
\draw[edge] (V1_1) -- node[edgeLabel] {$4$} (V5_2);
\draw[edge] (V3_2) -- node[edgeLabel] {$5$} (V2_1);
\draw[edge] (V4_1) -- node[edgeLabel] {$6$} (V2_1);
\draw[edge] (V2_1) -- node[edgeLabel] {$7$} (V5_1);
\draw[edge] (V5_3) -- node[edgeLabel] {$7$} (V2_1);
\draw[edge] (V3_1) -- node[edgeLabel] {$8$} (V4_1);
\draw[edge] (V4_1) -- node[edgeLabel] {$8$} (V3_2);
\draw[edge] (V5_2) -- node[edgeLabel] {$9$} (V3_1);
\draw[edge] (V3_2) -- node[edgeLabel] {$9$} (V5_3);


% Draw the vertices
\vertexLabelR{V1_1}{left}{$1$}
\vertexLabelR{V2_1}{left}{$2$}
\vertexLabelR{V3_1}{left}{$3$}
\vertexLabelR{V3_2}{left}{$3$}
\vertexLabelR{V4_1}{left}{$4$}
\vertexLabelR{V5_1}{left}{$5$}
\vertexLabelR{V5_2}{left}{$5$}
\vertexLabelR{V5_3}{left}{$5$}

\end{tikzpicture}

                        \end{scope}
                    \end{tikzpicture}
 
