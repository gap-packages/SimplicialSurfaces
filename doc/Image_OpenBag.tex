% This is the tikz-code for an open bag
% 
% The optional parameter is "\labels". If that is defined the bag will
% be labelled.

\begin{tikzpicture}
    \def\scale{1.7}        

    % Custom method to check whether something is visible
    \def\check#1{
        \ifdefined \labels
            #1
        \fi
    } 

    % Define the coordinates (with coloured labels)
    \coordinate [label={[\vertexColor]below:\check{1}}] (A) at (0,0);
    \coordinate [label={[\vertexColor]right:\check{2}}] (B) at (\scale,2*\scale);
    \coordinate [label={[\vertexColor]left:\check{3}}] (C) at (-1*\scale,2*\scale);


    % Fill faces	
    % We use a vertical shading to make it obvious which face is in the background
    % This is important since the reader otherwise might fall prey to an optical illusion
    % and confuse front and back. Since this would change the incidence geometry, this has
    % to be avoided at all costs.
    %
    % Since the generation of svg-images does not support shadings, we will replace the
    % shading by a darker fill-in in the HTML-version.
    \if\texforht
        \filldraw[fill=\faceColor!80!black]
    \else
        \filldraw[bottom color=\faceColor!53!black, top color=\faceColor]
    \fi
        (B) to[bend right=45] node[above, near start]{\check{3}} (C) to[bend right=45] (B);

    \filldraw[face] (A) -- node[right, near end]{\check{1}} (B) 
            to[bend left=45] node[above, near end]{\check{4}} (C) 
            -- node[left, near start]{\check{2}} cycle;
			
    
    % Draw face labels
    \node at ($1/2*(A)+1/4*(B)+1/4*(C)$) {\check{I}};
    \node at ($1/2*(B)+1/2*(C)$) {\check{II}};


    % Draw vertices
    \foreach \x in {A,B,C}
	\fill[vertex] (\x) circle (\vSize); 

\end{tikzpicture}
