% This is a picture of three triangles in a line, connected by edges.
% It relies on the styles that are defined for the general documentation

% It furthermore allows to turn on/off the labelling of vertices, edges and faces
% To turn on the labels, define \vertexLabels, \edgeLabels or \faceLabels before the start of this code
% WARNING: You should probably call this file like
% {
%   \def\vertexLabels{1} % if you want labels
%   \input{...}
% }
% Otherwise the labels will be switched on universally!

\begin{tikzpicture}[edgeLabel/.style={fill=red!10!white}]
    \coordinate [label=below:\ifdefined \vertexLabels 1 \fi] (A) at (0,0);
    \coordinate [label=below:\ifdefined \vertexLabels 3 \fi] (B) at (3,0);
    \coordinate [label=above:\ifdefined \vertexLabels 2 \fi] (C) at (1.5,2);
    \coordinate [label=above:\ifdefined \vertexLabels 4 \fi] (D) at ($(B)+(C)$);
    \coordinate [label=below:\ifdefined \vertexLabels 5 \fi] (E) at ($2*(B)$);

    % Draw a face with vertices 1,2,3 and label 4 in the middle
    \def\drawFace#1#2#3#4{
        \filldraw[face] (#1) -- (#2) -- (#3) -- cycle;
        \node at ($1/3*(#1)+1/3*(#2)+1/3*(#3)$) {#4};
    }
                
    \drawFace{A}{B}{C}{\ifdefined \faceLabels I \fi}
    \drawFace{B}{C}{D}{\ifdefined \faceLabels II \fi}
    \drawFace{B}{D}{E}{\ifdefined \faceLabels III \fi}

    \ifdefined \edgeLabels
        \node[edgeBackground] at ($1/2*(A)+1/2*(B)$) {2};
        \node[edgeBackground] at ($1/2*(A)+1/2*(C)$) {1};
        \node[edgeBackground] at ($1/2*(C)+1/2*(B)$) {3};
        \node[edgeBackground] at ($1/2*(D)+1/2*(B)$) {5};
        \node[edgeBackground] at ($1/2*(E)+1/2*(B)$) {6};
        \node[edgeBackground] at ($1/2*(C)+1/2*(D)$) {4};
        \node[edgeBackground] at ($1/2*(D)+1/2*(E)$) {7};
    \fi
\end{tikzpicture}
